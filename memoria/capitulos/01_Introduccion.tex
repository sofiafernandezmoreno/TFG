\chapter{Introducción} 

El objetivo de este proyecto es el completo despliegue de la herramienta \textbf{Naemon} a partir de contenedores \textbf{Docker}, apoyándonos de la creación de un sistema el cual analizaremos más adelante.

\section{Objetivos}

Para poder realizar el despliegue se va estructurar el proyecto de la siguiente forma mediante una serie de objetivos:
\begin{itemize}
	\item Revisión bibliográfica del estado del arte. Este objetivo pretende explorar el concepto de monitorización sobre cualquier punto objetivo, como puede ser una red o una aplicación, en nuestro caso llegaremos a la realización del modelado de un sistema como puede ser WordPress, para poder adquirir el conocimiento para plantearnos entre varias herramientas a utilizar, llegar a la elegida por este presente proyecto, la herramienta conocida como Naemon.
	\item Profundizaremos el conocimiento de Naemon a través de su despliegue utilizando el conjunto de tecnologías conocidas como contenedores, aprovechando para introducir así este concepto realizando una comparación con otros servicios.
	\item Se creará un sistema de cargas sintéticas, utilizando herramientas de análisis y mediciones del desempeño de varios servicios establecidos en este proyecto, para así obtener el propósito de crear una concurrencia real, llegando a un sistema final de comprobación.
	\item Realización de representación de la carga con la recopilación de datos obtenidos mediante complementos ofrecidos por la herramienta Naemon.	
\end{itemize}

\section{Planificación del proyecto}
\begin{itemize}
	\item \textbf{T1}: Primer trimestre. Septiembre - Diciembre 2018
	\item \textbf{T2}: Segundo trimestre. Enero - Marzo 2019
	\item \textbf{T3}: Tercer trimestre. Abril - Junio 2019
	\item \textbf{T4}: Cuarto trimestre. Junio - Septiembre 2019
\end{itemize}

\begin{ganttchart}[%Specs
	x unit = 1.2cm,  %<---------------------- New x unit 
	y unit title=0.5cm,
	y unit chart=0.5cm,
	vgrid, hgrid,
	title height=1,
	%     title/.style={fill=none},
	title label font=\bfseries\footnotesize,
	bar/.style={fill=blue},
	bar height=0.7,
	%   progress label text={},
	group right shift=0,
	group top shift=0.7,
	group height=.3,
	group peaks width={0.2},
	inline]{1}{4}
	%labels
	
	\gantttitle[]{2018-2019}{4} \\                 % title                      % title 3
	\gantttitle{T1}{1}
	\gantttitle{T2}{1}
	\gantttitle{T3}{1}
	\gantttitle{T4}{1}\\
	%\gantttitle{Marzo}{1}\\
	
	% Setting group if any
	
	\ganttgroup[inline=false]{Estado técnica}{1}{2}\\ 
	
	\ganttbar[progress=100,inline=false]{\textit{Aprendizaje}}{1}{2}\\
	\ganttbar[progress=100,inline=false]{\textit{Tecnologías}}{1}{2}\\

	
	\ganttgroup[inline=false]{Despliegue}{1}{2} \\
	
	\ganttbar[progress=100,inline=false]{\textit{Dockerfile}}{1}{2} \\	
	\ganttbar[progress=100,inline=false]{\textit{Naemon}}{1}{2} \\
	\ganttbar[progress=100,inline=false]{\textit{Thruk}}{1}{2} \\
	\ganttbar[progress=100,inline=false]{\textit{Locust}}{1}{2} \\
	\ganttbar[progress=100,inline=false]{\textit{Docker-Compose}}{1}{2} \\
	
	\ganttgroup[inline=false]{Prueba de carga}{2}{3} \\	
	\ganttbar[progress=100,inline=false]{\textit{Test}}{2}{3} \\
	
	\ganttbar[progress=100,inline=false]{\textit{Gráficos}}{2}{3} \\ç

	
	
	\ganttgroup[inline=false]{Fase final}{3}{4} \\ 
	
	 
	%\ganttbar[progress=50,inline=false, bar progress label node/.append style={below left= 10pt and 7pt}]{Task B}{3}{4} \\ \\
	\ganttbar[progress=100,inline=false]{Experimentos}{3}{3}\\ 
	\ganttbar[progress=100,inline=false]{Resultados}{3}{3} \\
	\ganttbar[progress=100,inline=false]{Conclusiones}{4}{4} \\ 

	\ganttbar[progress=100,inline=false]{Documentación}{4}{4} \\ 
	
\end{ganttchart}
\newpage
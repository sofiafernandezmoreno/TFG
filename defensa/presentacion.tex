\documentclass{beamer}

%% Configuración de la presentación
\mode<presentation> {

  \usetheme{Warsaw}

  
  \usecolortheme{beaver}
 
}

%% Fuentes de tamaño arbitrario
\usepackage{lmodern}
\usepackage{amsfonts}
\usepackage{dsfont}


%% Gráficos
\usepackage{graphicx} % Allows including images
\usepackage{booktabs} % Allows the use of \toprule, \midrule and \bottomrule in tables

%%% Castellano.
% noquoting: Permite uso de comillas no españolas.
% lcroman: Permite la enumeración con numerales romanos en minúscula.
% fontenc: Usa la fuente completa para que pueda copiarse correctamente del pdf.
\usepackage[english,spanish,es-noquoting,es-lcroman]{babel}
\usepackage[utf8]{inputenc}
\usepackage[T1]{fontenc}
\selectlanguage{spanish}

\usepackage{tikz}
\usepackage{tikz-cd}
\usepackage{tikz-3dplot}

\usepackage{verbatim}
\usetikzlibrary{arrows,shapes}
\usepackage{amsthm}
\usepackage{accents}
\usepackage{amsmath,amssymb,lmodern}
\usepackage{fontawesome}
\usepackage{setspace}

% Definitions
\theoremstyle{plain}
\newtheorem{thm}{Teorema}
\theoremstyle{definition}
\newtheorem{defn}[thm]{Definici\'{o}n}
\theoremstyle{plain}
\newtheorem{prop}[thm]{Proposici\'{o}n}
\theoremstyle{definition}
\theoremstyle{remark}
\newtheorem{rem}[thm]{Nota}
\theoremstyle{definition}
\newtheorem{lem}[thm]{Lema}
\newtheorem{cor}[thm]{Corolario}
\newtheorem{ejemplo}[thm]{Ejemplo}
\newtheorem{ejemplos}[thm]{Ejemplos}
% Counter

\newcounter{saveenumi}
\newcommand{\seti}{\setcounter{saveenumi}{\value{enumi}}}
\newcommand{\conti}{\setcounter{enumi}{\value{saveenumi}}}

% Sections
\AtBeginSection{\frame{\sectionpage}}
\newtranslation[to=spanish]{Section}{Sección}

%comandos para agilizar la escritura de simbolos frecuentes
\newcommand{\sphere}{\mathds{S}^{d-1}}
\newcommand{\esfera}{\mathds{S}^{2}}
\newcommand{\orto}{\mathbb{O}^d}
\newcommand{\R}{\mathds{R}^d}
\newcommand{\spharm}{\mathds{Y}^d_n}
\newcommand{\sphint}{\int_{\sphere}}

\defbeamertemplate{section page}{mine}[1][]{%
  \begin{centering}
    {\usebeamerfont{section name}\usebeamercolor[fg]{section name}#1}
    \vskip1em\par
    \begin{beamercolorbox}[sep=12pt,center]{part title}
      \usebeamerfont{section title}\insertsection\par
    \end{beamercolorbox}
  \end{centering}
}

%----------------------------------------------------------------------------------------
%	TÍTULO
%----------------------------------------------------------------------------------------

\title[]{Análisis y monitorización de aplicaciones a través de plugins con Naemon} % The short title appears at the bottom of every slide, the full title is only on the title page

\author{Sofía Fernández Moreno} % Your name
\institute[UGR] % Your institution as it will appear on the bottom of every slide, may be shorthand to save space
{
  Universidad de Granada \\ % Your institution for the title page
  % Your email address
}
\date{Septiembre de 2019} % Date, can be changed to a custom date




\begin{document}
% Spanish
\selectlanguage{spanish}
% Para tikz
\pgfdeclarelayer{background}\theoremstyle{definition}
\pgfsetlayers{background,main}
% Secciones
\setbeamertemplate{section page}[mine]

%% Diapositiva de título.
\frame{\titlepage}

%% Diapositiva de contenidos.
% Throughout your presentation, if you choose to use \section{} and \subsection{} commands,
% these will automatically be printed on this slide as an overview of your presentation
\begin{frame}
  \frametitle{Contenidos} % Table of contents slide, comment this block out to remove it
  \tableofcontents
\end{frame}


%----------------------------------------------------------------------------------------
%	PRESENTACIÓN
%----------------------------------------------------------------------------------------

%------------------------------------------------
\section{Estado del arte} % Sections can be created in order to organize your presentation into discrete blocks, all sections and subsections are automatically printed in the table of contents as an overview of the talk
%------------------------------------------------
\begin{frame}
	
	\frametitle{Objetivos}
	\begin{itemize}
		\item Introducir concepto de monitorización
		\item Comparar diferentes herramientas de monitorización
		\item Introducción a la herramienta Naemon
	\end{itemize}
	
\end{frame}
\subsection{¿Qué es la monitorización?}
\begin{frame}
	\frametitle{¿Qué es la monitorización?}
	\begin{defn}
		La \textbf{monitorización} es aquella en la que se toman las medidas preventivas y
		consecuentes con la información que se obtienen de todos los dispositivos
		que se encuentran conectados a una red, para evitar posibles eventos que
		hacen que se interrumpan el correcto funcionamiento de alguno de ellos
	\end{defn}
Dentro de este concepto es importanteaplicar el uso del protocolo \textbf{SNMP} (Simple Network
Management Protocol).
\begin{defn}
	\textbf{SNMP} permite el intercambio de información amplia entre los diferentes dispositivos de red mediante consultas de forma remota (polling) y mediante mensajes basándose en eventos (traps).
\end{defn}

	
\end{frame}

\subsection{Comparativa herramientas de monitorización}
\begin{frame}
	\frametitle{Comparativa herramientas de monitorización}
	Incluir imagen de todas	
\end{frame}

\subsection{Naemon}
\begin{frame}
	\frametitle{Funcionamiento}
	
\end{frame}
\begin{frame}
	\frametitle{Archivo de configuración principal}
		
\end{frame}

\begin{frame}
	\frametitle{Tipos de objetos}
	
\end{frame}

\begin{frame}
	\frametitle{Interfaz GUI: Thruk}
	
\end{frame}


\section{Realización del despliegue de Naemon} % Sections can be created in order to organize your presentation into discrete blocks, all sections and subsections are automatically printed in the table of contents as an overview of the talk
%------------------------------------------------
\begin{frame}
	
	\frametitle{Objetivos}
	\begin{itemize}
		\item Introducir entorno en el que se desarrolla el despliegue
		\item Realizar despliegue de Naemon mediante una nueva imagen
		\item Realizar despliegue mediante orquestación estática (Docker-Compose)
		\item Enlazar en el despliegue el sistema a analizar
	\end{itemize}
	
\end{frame}

\subsection{Entorno de desarrollo}
\begin{frame}
	\frametitle{Docker Engine}
	
\end{frame}
\subsection{Desarrollo de despliegue}
\begin{frame}
	\frametitle{Dockerfile con imagen de Naemon}
	
\end{frame}
\begin{frame}
	\frametitle{Orquestación estática de aplicaciones}
	\begin{defn}
		La \textbf{orquestación estática} es aquella que el sistema requiere una configuración más
		manual de los recursos y no permite el escalado de forma muy eficiente.
	\end{defn}
\end{frame}
\begin{frame}
	\frametitle{Docker-Compose}
\end{frame}
\section{Pruebas de carga} % Sections can be created in order to organize your presentation into discrete blocks, all sections and subsections are automatically printed in the table of contents as an overview of the talk
%------------------------------------------------
\begin{frame}
	
	\frametitle{Objetivos}
	\begin{itemize}
		\item Introducir tipos de pruebas de rendimiento
		\item Comparar herramientas de pruebas de carga y seleccionar la opción elegida
		\item Introducir el funcionamiento de la herramienta Locust
		\item Comprender la forma de configurar Locust 
	\end{itemize}
	
\end{frame}

\subsection{Pruebas de rendimiento}
\begin{frame}
	\frametitle{Tipos de pruebas}
	
\end{frame}
\begin{frame}
	\frametitle{Elección a realizar}
	
\end{frame}

\subsection{Comparativa de herramientas}
\begin{frame}
	\frametitle{Comparativa de herramientas}
	
\end{frame}

\subsection{Locust}
\begin{frame}
	\frametitle{Funcionamiento de Locust}
	
\end{frame}
\begin{frame}
	\frametitle{Locustfile
\end{frame}
\section{Pruebas de carga en un sistema} % Sections can be created in order to organize your presentation into discrete blocks, all sections and subsections are automatically printed in the table of contents as an overview of the talk
%------------------------------------------------
\begin{frame}
	
	\frametitle{Objetivos}
	\begin{itemize}
		
		\item Configurar archivo Locustfile y enlazarlo en la orquestación
		\item Realizar las pruebas de carga de WordPress en Locust
		\item Realizar prueba de monitorización del sistema WordPress en Naemon
		\item Aplicar el desarrollo de un plugin en el host(sistema WordPress) y servicios añadidos a Naemon.		
	\end{itemize}
	
\end{frame}

\subsection{Locust en Docker-Compose}
\begin{frame}
	\frametitle{Funcionamiento de Locust}
	
\end{frame}
\begin{frame}
	\frametitle{Locustfile
	\end{frame}

\section{Modelado de las pruebas} % Sections can be created in order to organize your presentation into discrete blocks, all sections and subsections are automatically printed in the table of contents as an overview of the talk
%------------------------------------------------
\begin{frame}
	
	\frametitle{Objetivos}
	\begin{itemize}
		\item Explicar que es la carga de trabajo
		\item Obtener resultados de la carga de trabajo mediante herramienta PNP4Nagios
		\item Enlazar PNP4Nagios a la imagen creada en el despliegue de Naemon
		\item Integrar PNP4Nagios en Thruk
		\item Exportar los resultados en CSV			
	\end{itemize}
	
\end{frame}

\section{Conclusiones y futuros trabajos}

\begin{frame}{}{}
	\Huge{\centerline{Gracias por su atención}}
	\centerline{\Huge{\raisebox{-.25\height}\faGithub}~\large{\href{https://github.com/sofiafernandezmoreno/TFG}{\alert{sofiafernandezmoreno/TFG}}}}
	
	
	

\end{frame}
\end{document}